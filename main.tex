\documentclass{article}
\usepackage{graphicx} % Required for inserting images

\title{FZU: DFT Calculattions}
\author{many}
\date{March 2024}

\begin{document}

\maketitle

\section{Tasks }:


\begin{itemize}
    \item Check the convergence of the ground state energies for the uu nn dd (ferrimagnetic), ud nn ud (antiferromagnetic)  and uu nn uu (ferromagnetic) configuration with respect to the number of k-points. (LDA and GGA)
    \item Plot please E(\#k), M(\#k)  t(\#k)!
    \item Extract the minimum necessary k-points.
\end{itemize}




\subsection{Calculate Ground state energies and magnetic momenta for the following 
configurations and the LDA and GGA functionals:}
(see Table \ref{tab:my_label})

\begin{table}
    \centering
    \begin{tabular}{ccccccc}
         &  &  LDA&  &  &  GGA& \\
         &  Energy&  mag. Moment&    $\Delta E$&  Energy&  mag. Moments&   $\Delta E$\\
         uu nn dd (ferrimagnetic) &  &  &  &  &  |& \\
         ud nn ud& identical to ud nn du   &  &  &  &  & \\
         ud nn du &  &  &  &  &  & \\
         du nn du &  &  &  &  &  & \\
         du  nn ud &  &  &  &  &  & \\
        dd nn dd (ferromagnetic)&  &  &  &  &  & \\
        uu nn uu (ferromagnetic)&  &  &  &  &  & \\
    \end{tabular}
    \caption{Overview of necessary runs. Use the (suposedly converged) k-point number extracted above. $\Delta E$ refers to the energy difference of each configuration to the ground state energy of the respective lowest configuration in energy. For GGA we expect the ferrimagnetic configuration to be the lowest in energy and $\Delta E = 15 mRydberg$ for ud nn ud (AFM 1). }
    \label{tab:my_label}
\end{table}

	 


Nice sticks and balls plots with magnetic momenta depicted?

\subsection{Only once (?)}
\begin{enumerate}
    \item DOS:	calculate DOS for GGA and LDA (ferrimagnetic) 
	\item DOSplot for each and $\Delta$ DOS 
 \item Analyse convergence of each run and try to identify "suspicious" runs.?
 \ \end{enumerate}
\end{document}
