\documentclass{article}

\usepackage{a4wide}

\def\mn2sb{Mn$_2$Sb}

\title{The four Cu$_2$Sb-type compounds: Mn, Co and As, Sb}

\author{Satya, Ondra, Christian, Karel}

\date{Mar19, 2024}

\begin{document}

\maketitle

Antiferromagnetic spintronics~\cite{Baltz:2018_a} needs thin layers...

Motivated by the utility of CuMnAs~\cite{Veis:2018_a} in this field,
we look at other materials of the same structure that could possibly
be grown epitaxially. We begin with \mn2sb which is, in the ground
state, ferrimagnetic. In bulk, this is a well known material.
Interestingly, by alloying Sb with As, it transforms into an
antiferromagnet~\cite{Shirakawa:1976_a} (which can, however, return to
the ferrimagnetic state upon increasing temperature). Simple DFT
yields the following energetics (in mRy/f.u.):

\begin{center}
  \begin{tabular}{c|cc}
    & $\Delta E$ & remark \\ \hline
    ferri & 0 \\
    FM    & 15.4 & unconverged \\
    AFM1  & 12.6 & ...\\
    AFM2  && ... the same
  \end{tabular}
\end{center}

Magnetic moments (mmoms) depend only weakly on the type of magnetic order: 
Mn mmom at 2a Wyckoff position is smaller (about 2.9 $\mu_B$) and the other
(at 2c) is about 3.7 $\mu_B$.

\begin{thebibliography}{99}

\bibitem{Baltz:2018_a} RMP 90, 015005

\bibitem{Veis:2018_a} PRB 97, 125109

\bibitem{Shirakawa:1976_a} Shirakawa and Ido,
  {\tt doi: 10.1143/JPSJ.40.666}
  
\end{thebibliography}

\end{document}

